\documentclass{article}


\usepackage[utf8]{inputenc}
\usepackage[T1]{fontenc}
\usepackage[USenglish]{babel}
\usepackage{amsmath,amssymb}
\usepackage{fullpage}
\usepackage{todonotes}


\makeatletter
\providecommand*{\diff}%
   {\@ifnextchar^{\DIfF}{\DIfF^{}}}
\def\DIfF^#1{%
   \mathop{\text{\mathstrut d}}%
      \nolimits^{#1}\gobblespace}
\def\gobblespace{%
      \futurelet\diffarg\opspace}
\def\opspace{%
   \let\DiffSpace\!%
   \ifx\diffarg(%
      \let\DiffSpace\relax
   \else
      \ifx\diffarg[%
         \let\DiffSpace\relax
      \else
         \ifx\diffarg\{%
            \let\DiffSpace\relax
         \fi\fi\fi\DiffSpace}
\makeatother
\newcommand{\intd}{\diff}%             differential "d"
\newcommand{\defeq}{\mathrel{:=}}%     is per definition equal to
\newcommand{\tp}{\top}%                transpose
\newcommand{\kron}{\otimes}%           Kronecker product
\newcommand{\setB}{\mathbb{B}}%        Booleans
\newcommand{\setR}{\mathbb{R}}%        real numbers
\newcommand{\expe}{\mathrm{e}}%        Eulers constant

\newcommand{\vectorfont}[1]{\boldsymbol{#1}}%
\newcommand{\greekvectorfont}[1]{\boldsymbol{#1}}%
\newcommand{\matrixfont}[1]{\mathbf{#1}}%


\newcommand{\bvec}{\vectorfont{b}}
\newcommand{\fvec}{\vectorfont{f}}
\newcommand{\gvec}{\vectorfont{g}}
\newcommand{\hvec}{\vectorfont{h}}
\newcommand{\uvec}{\vectorfont{u}}
\newcommand{\yvec}{\vectorfont{y}}

\newcommand{\tildeyvec}{\vectorfont{\tilde{y}}}

\newcommand{\barxvec}{\vectorfont{\bar{x}}}

\newcommand{\dyvec}{\vectorfont{\dot{y}}}
%
\newcommand{\Phivec}{\greekvectorfont{\Phi}}
%
\newcommand{\bfrakvec}{\vectorfont{\mathfrak{b}}}
\newcommand{\ffrakvec}{\vectorfont{\mathfrak{f}}}
\newcommand{\ufrakvec}{\vectorfont{\mathfrak{u}}}
\newcommand{\xfrakvec}{\vectorfont{\mathfrak{x}}}
\newcommand{\yfrakvec}{\vectorfont{\mathfrak{y}}}
%
\newcommand{\barffrakvec}{\vectorfont{\overline{\mathfrak{f}}}}
\newcommand{\barxfrakvec}{\vectorfont{\overline{\mathfrak{x}}}}
%
\newcommand{\nullvec}{\greekvectorfont{0}}
\newcommand{\lbvec}{\vectorfont{l\negthinspace b}}
\newcommand{\ubvec}{\vectorfont{u\negthinspace b}}
\newcommand{\lbfrakvec}{\vectorfont{\mathfrak{l\negthinspace b}}}
\newcommand{\ubfrakvec}{\vectorfont{\mathfrak{u\negthinspace b}}}
%
\newcommand{\lbbarfrakvec}{\vectorfont{\overline{\mathfrak{l\negthinspace b}}}}
\newcommand{\ubbarfrakvec}{\vectorfont{\overline{\mathfrak{u\negthinspace b}}}}
%
\newcommand{\Bmat}{\matrixfont{B}}
\newcommand{\Hmat}{\matrixfont{H}}
\newcommand{\Imat}{\matrixfont{I}}%       identity matrix
\newcommand{\Smat}{\matrixfont{S}}

\newcommand{\Afrakmat}{\matrixfont{\mathfrak{A}}}

\newcommand{\barAfrakmat}{\matrixfont{\overline{\mathfrak{A}}}}

\newcommand{\Nullmat}{\matrixfont{0}}



\begin{document}

\begin{center}
	{\Large\textbf{Implementation Details for \texttt{mi\_cp\_linprog} (in \texttt{oc.py})}}
\end{center}

We have a linear \textbf{m}ixed \textbf{i}nteger optimal control problem of the form
\begin{subequations}
\begin{align}
\min_{\uvec(\cdot), \yvec (\cdot), \barxvec(\cdot)} &\int_{t_0}^{t_{\mathrm{end}}} \expe^{-\varphi \cdot t}\cdot \Phivec_1^{\tp} \cdot\yvec(t) \intd t + \Phivec_2^{\tp}\cdot \yvec(t_0) + \Phivec_3^{\tp} \cdot \yvec(t_{\mathrm{end}})
\label{eq:OC_problem:obj}
\\
\text{s.\,t.} \quad
\dyvec(t) &= \Smat_2 \cdot \uvec(t) + \Smat_4 \cdot \yvec(t)
\label{eq:OC_problem:dyn}
\\
\nullvec &= \Smat_1 \cdot \uvec(t)
\label{eq:OC_problem:qssa}
\\
\lbvec &\leq \uvec (t)\leq \ubvec
\label{eq:OC_problem:flux_bounds}
\\
\Hmat_{\yvec} \cdot \yvec(t)  + \Hmat_{\uvec}\cdot \uvec(t)&\leq \hvec
\label{eq:OC_problem:mixed}
\\
\nullvec &\leq \yvec(t)
\label{eq:OC_problem:pos}
\\
\Hmat_{\setB,\yvec}\cdot \yvec(t) + \Hmat_{\setB,\uvec}\cdot \uvec(t) + \Hmat_{\setB,\barxvec} \cdot \barxvec(t) &\leq \hvec_{\setB}
\label{eq:OC_problem:fullmixed}
\\
\Bmat_{\yvec_0} \cdot \yvec(t_0) + \Bmat_{\yvec_{\mathrm{end}}} \cdot \yvec(t_{\mathrm{end}}) &= \bvec_{\mathrm{bndry}}
\label{eq:OC_problem:bndry}
\\
\yvec (t) &\in \setR^{n_{\yvec}}\,,~\uvec(t) \in \setR^{n_{\uvec}}
\label{eq:OC_problem:cont}
\\
\barxvec(t) &\in \setB^{n_{\barxvec}}
\label{eq:OC_problem:boole}
\end{align}
\end{subequations}
and want to approximate its solution using a \textbf{c}omplete \textbf{p}arameterization approach with midpoint rule for the dynamics and trapezoidal rule for the Lagrange part of the objective.
The result will be formulated as a large linear program
\begin{equation}
\begin{split}
\min_{\xfrakvec, \barxfrakvec} &\;\ffrakvec^{\tp} \cdot \xfrakvec + \barffrakvec^{\tp} \cdot \barxfrakvec\\
\text{s.\,t.}\quad
\Afrakmat_{\leq} \cdot \xfrakvec &\leq \bfrakvec_{\leq}
\\
\Afrakmat_{=} \cdot \xfrakvec &\leq \bfrakvec_{=}
\\
\lbfrakvec &\leq \xfrakvec \leq \ubfrakvec
\\
\Afrakmat_{\setB,\leq} \cdot \xfrakvec + \barAfrakmat_{\setB,\leq} \cdot \barxfrakvec &\leq \bfrakvec_{\setB,\leq}
\\
\xfrakvec &\in \setR^{n_{\xfrakvec}}\,,~\barxfrakvec \in \setB^{n_{\barxfrakvec}}
\end{split}
\label{eq:final_LP}
\end{equation}

The first question concerns the choice of the time grid for the Boolean variables.
We go with the original proposal and define them on the shifted time grid
\[
\barxvec_{m+1/2} \approx \barxvec(t_{m+1/2}^s)\,~_(m=0,1,\ldots,N-1)
\] 
(even though this might be debatable and we could maybe later add a feature to choose this).


Building most parts of the matrices is equivalent to the purely continuous part. 
All we need to add is structure for \eqref{eq:OC_problem:fullmixed}.
This is equivalent to the way, these equations are obtained for \eqref{eq:OC_problem:mixed}.
For the sake of completeness:
We require the discretizations, to hold on the shifted time grid and use the equations
\[
\Hmat_{\setB,\yvec} \cdot \frac{\yvec_{m}+ \yvec_{m+1}}{2} + 
\Hmat_{\setB, \uvec}\cdot \uvec_{m+1/2} + 
\Hmat_{\setB, \barxvec} \cdot \barxvec_{m+1/2}
\leq \hvec_{\setB}
\]
which are being inflated to the variables $\xfrakvec$ and $\barxfrakvec$, resp.
\end{document}


We start with the construction of two time grids, one (called $(t_i)_{i=1}^N$) for the dynamic variables $\yvec(\cdot)$ and one (called $(t_m^{\mathrm{s}})_{m=1/2}^{(N-1)+1/2}$) for the controls $\uvec(\cdot)$.
The user-defined parameter $N \geq 2$ defines the number of time steps.
In the program, these will be referenced as \texttt{tt} and \texttt{tt\_shift}.
The entries of this time grids are defined via
\begin{equation*}
\begin{split}
t_m &\defeq t_0 + m \cdot \Delta_t \,,~(m = 0,1,\ldots, N)\,,
\\
t_{m+1/2}^{\mathrm{s}} &\defeq \frac{t_m + t_{m+1}}{2}\,,~(m=0,1,\ldots,N-1)\,,
\end{split}
%\label{eq:}
\end{equation*}
where the time step size $\Delta_t$ is given by $\Delta_t \defeq \frac{t_{\mathrm{end}} - t_0}{N}$.
We introduce approximations to the dynamic and discrete variables in a straightforward way
\[
\yvec_m \approx \yvec (t_m)\,,~(m=1,2,\ldots,N)\,\qquad
\uvec_{m+1/2} \approx \uvec (t_{m+1/2}^{\mathrm{s}})\,,~(m=1,2,\ldots,N-1)\,.
\]
So, all approximations will be collected in two stacked vectors
\begin{equation*}
\begin{split}
\yfrakvec &\defeq (\yvec_0^{\tp}, \yvec_1^{\tp} , \ldots , \yvec_{N}^{\tp})^{\tp}\,,
\\
\ufrakvec &\defeq (\uvec_{1/2}^{\tp}, \uvec_{1+1/2}^{\tp}, \ldots , \uvec_{(N-1)+1/2}^{\tp})^{\tp}\,.
\end{split}
%\label{eq:}
\end{equation*}
The vector $\xfrakvec$ in \eqref{eq:final_LP} consists of all the discrete approximations $\xfrakvec = (\yfrakvec^{\tp},\ufrakvec^{\tp})^{\tp}$.


% %%%%%%%%%%%%%%%%%%%%%%%%%%%%%%%%%%%%%%%%%%%%%%%%%%%%%%%%%%%%%%%%%%%%%%%%%%%%%%
\subsection*{Formulation of the Linear Program}

In case that (some of) the system matrices are constant in time (and we use an equidistant time grid), the LP-problem matrices $\Afrakmat_{\leq}$ and $\Afrakmat_{=}$ can efficiently be implemented using the Kronecker product:
\begin{multline*}
\kron \colon \setR^{m_A \times n_A} \times \setR^{m_B \times n_B} \to \setR^{m_A \cdot m_B \times n_A \cdot n_B}\\
\text{with}\quad
\begin{pmatrix} a_{11} & \cdots & a_{1,n_A} \\ \vdots \\ a_{m_A, 1} & \cdots & a_{m_A, n_A}\end{pmatrix}
\kron B \defeq 
\begin{pmatrix} a_{11}\cdot B & \cdots & a_{1,n_A}\cdot B \\ \vdots \\ a_{m_A, 1}\cdot B & \cdots & a_{m_A, n_A} \dot B\end{pmatrix}\,.
\end{multline*}
For the LP-problem matrices and vectors we will treat the part belonging to the dynamic variables and the ones belonging to the control variables separately.
% ______________________________________________________________________________
\subsubsection*{Objective}

The Lagrange part will be approximated using the trapezoidal rule (on the dynamic time grid), i.\,e.:
\[
\int_{t_0}^{t_{\mathrm{end}}} f(t) \intd t
\approx \Delta_t \cdot
\left(\frac{f(t_0)}{2} + f(t_1) + f(t_2) + \cdots + f(t_{N-1} + \frac{f(t_N)}{2}) \right)
\]
That means, that the according part of the objective vector just collects the $\Phivec_1$ entries like
\[
\ffrakvec_{\yfrakvec,\mathrm{Lagrange}} \defeq \Delta_t \cdot 
\begin{pmatrix} \frac{\expe^{-\varphi t_0} \cdot\Phivec_1^{\tp}}{2} & \expe^{-\varphi t_1} \cdot\Phivec_1^{\tp} & \cdots & \expe^{-\varphi t_{N-1}} \cdot\Phivec_1^{\tp} & \frac{\expe^{-\varphi t_N} \cdot\Phivec_1^{\tp}}{2}\end{pmatrix}^{\tp}
\]
For the Mayer part, we just have to add terms for the start and end point:
\[
\ffrakvec_{\yfrakvec,\mathrm{Mayer}} \defeq 
\begin{pmatrix}
	\Phivec_2^{\tp} & \nullvec^{\tp} & \Phivec_3^{\tp}
\end{pmatrix}^{\tp}\,,
\]
such that the (dynamic part of the) objective vector of the resulting LP is given by
\[
\ffrakvec_{\yfrakvec} = \ffrakvec_{\yfrakvec,\mathrm{Lagrange}} + \ffrakvec_{\yfrakvec,\mathrm{Mayer}}\,.
\]

% ______________________________________________________________________________
\subsubsection*{Dynamics}

For the dynamics, we use the midpoint rule:
\[
\dyvec(t) = \fvec(t, \yvec(t), \uvec(t))\qquad\Rightarrow\qquad
\yvec_{m+1} = \yvec_{m} + \Delta_t \cdot \fvec(t_{m+1/2}, \tildeyvec_{m+1}^{(1)}, \uvec_{m+1/2})\,\quad (m=0,1,\ldots,N-1)\,,
\]
where $\tildeyvec_{m+1}^{(1)}$ is the according intermediate value for the dynamic variables (that does not (yet) play a role here).
In detail, the update equations for this scheme read
\[
\yvec_{m+1} = \yvec_m + \Delta_t \cdot \Smat_2 \cdot \uvec_{m+1/2}
\quad\Leftrightarrow
\yvec_m - \yvec_{m+1} + \Delta_t \cdot \Smat_2 \cdot \uvec_{m+1/2} = \nullvec\,,~(m = 0,1,\ldots,N-1)\,.
\]
In (large-) matrix form, all these equations read
{\small
\[
\begin{pmatrix}
\Imat_{n_{\yvec}} & -\Imat_{n_{\yvec}} & \Nullmat           & \cdots & \cdots & \Nullmat \\
\Nullmat          & \Imat_{n_{\yvec}}  & -\Imat_{n_{\yvec}} & \cdots & \cdots & \Nullmat \\
\vdots            &                    & \ddots             & \ddots &        & \vdots \\
\Nullmat          & \cdots             & \cdots             & &\Imat_{n_{\yvec}} & -\Imat_{n_{\yvec}}
\end{pmatrix}
\cdot
\begin{pmatrix}
	\yvec_0 \\ \yvec_1 \\ \vdots \\ \yvec_N
\end{pmatrix}
+
\begin{pmatrix}
	\Delta_t \cdot \Smat_2 & \Nullmat               & \cdots   & \Nullmat \\
	\Nullmat               & \Delta_t \cdot \Smat_2 & \cdots   & \Nullmat \\
	\vdots                 &                        & \ddots   & \vdots   \\
	\Nullmat               & \cdots                 & \Nullmat & \Delta_t \cdot \Smat_2
\end{pmatrix}
\cdot
\begin{pmatrix}
	\uvec_{1/2} \\ \uvec_{1+1/2} \\ \vdots \\ \uvec_{(N-1)+1/2}
\end{pmatrix}
=
\begin{pmatrix}
	\nullvec \\
	\nullvec \\
	\vdots \\
	\nullvec 
\end{pmatrix}
\]
}
With Kronecker-notation, this becomes
\[
\underbrace{
\begin{pmatrix}
	1      & -1     &  0     & \cdots & \cdots & 0 \\
	0      &  1     & -1     & \cdots & \cdots & 0 \\
	\vdots &        & \ddots & \ddots &        & \vdots \\
	0      & \cdots & \cdots &        & 1      & -1
\end{pmatrix}
\kron \Imat_{n_{\yvec}} 
}_{\Afrakmat_{\yvec,\mathrm{dyn}}} 
\cdot 
\underbrace{
\begin{pmatrix}
	\yvec_0 \\ \yvec_1 \\ \vdots \\ \yvec_N
\end{pmatrix}
}_{\yfrakvec}
+
\underbrace{
\Imat_{N} \kron \Delta_t \cdot \Smat_2
}_{\Afrakmat_{\uvec, \mathrm{dyn}}}
\cdot
\underbrace{
\begin{pmatrix}
	\uvec_{1/2} \\ \uvec_{1+1/2} \\ \vdots \\ \uvec_{(N-1)+1/2}
\end{pmatrix}
}_{\ufrakvec}
= \nullvec\,.
\]
% ______________________________________________________________________________
\subsubsection*{Control Constraints (QSSA rows)}

The constraints of the DAE system need to be fulfilled at the collocation points.
For the midpoint rule, this is exactly the shifted time grid:
\[
\nullvec = \gvec (t, \yvec(t), \uvec(t))
\qquad \Rightarrow\qquad
\nullvec = \gvec(t_{m+1/2}^{\mathrm{s}}, \tildeyvec_{m+1}^{(1)}, \uvec_{m+1/2})\,~(m = 0,1,\ldots,N-1)\,.
\]
Since we again have no dependency on $\yvec(\cdot)$ in the right-hand-side, this can simply be stated as
\[
\nullvec = \Smat_1 \cdot \uvec_{m+1/2}\,~(m = 0,1,\ldots,N-1)\,.
\]
In (large) matrix form:
\[
\Nullmat 
\cdot 
\begin{pmatrix}
	\yvec_0 \\ \yvec_1 \\ \vdots \\ \yvec_N
\end{pmatrix}
+
\begin{pmatrix}
	\Smat_1  & \Nullmat & \cdots & \Nullmat \\
	\Nullmat & \Smat_1  & \cdots & \Nullmat \\
	\vdots   &          & \ddots & \vdots   \\
	\Nullmat & \cdots   & \cdots & \Smat_1
\end{pmatrix}
\cdot 
\begin{pmatrix}
	\uvec_{1/2} \\ \uvec_{1+1/2} \\ \vdots \\ \uvec_{(N-1)+1/2}
\end{pmatrix}
= \nullvec
\]
and, again condensed using Kronecker-notation,
\[
\underbrace{\Nullmat}_{\Afrakmat_{\yvec,\mathrm{qssa}}} 
\cdot 
\underbrace{
\begin{pmatrix}
	\yvec_0 \\ \yvec_1 \\ \vdots \\ \yvec_N
\end{pmatrix}
}_{\yfrakvec}
+
\underbrace{
\Imat_{N} \kron \Smat_1
}_{\Afrakmat_{\uvec, \mathrm{qssa}}}
\cdot 
\underbrace{
\begin{pmatrix}
	\uvec_{1/2} \\ \uvec_{1+1/2} \\ \vdots \\ \uvec_{(N-1)+1/2}
\end{pmatrix}
}_{\ufrakvec}
= \nullvec\,.
\]

% ______________________________________________________________________________
\subsubsection*{Control Bounds}

The bounds on the control variables just need to be applied at very time step:
\[
\lbfrakvec_{\uvec}
= \begin{pmatrix} 	\lbvec^{\tp} & \lbvec^{\tp} & \cdots & \lbvec^{\tp} \end{pmatrix}^{\tp}
\,\quad
\ubfrakvec_{\uvec}
= \begin{pmatrix} 	\ubvec^{\tp} & \ubvec^{\tp} & \cdots & \ubvec^{\tp} \end{pmatrix}^{\tp}
\]
in order to implement
\[
\lbvec \leq \uvec_{m+1/2} \leq \ubvec \quad (m = 0, 1, \ldots,N-1)
\]
in the form $\lbfrakvec_{\uvec} \leq \ufrakvec \leq \ubfrakvec$.
% ______________________________________________________________________________
\subsubsection*{Mixed Constraints}

The mixed constraints require that we first take a closer look at the Runge-Kutta updates inherent in the midpoint rule.
The mixed constraints, just like the flux constraints, need to be enforced on the shifted time grid.
That means, an inequality constraint of the form
\[
\nullvec \leq \gvec_{\leq} (t, \yvec(t), \uvec(t))
\]
is discretized to
\begin{equation}
\nullvec = \gvec (t_{m+1/2}^{\mathrm{s}}, \tildeyvec_{m+1}^{(1)}, \uvec_{m+1/2})\,,~(m = 0,1,\ldots,N-1)\,.
\label{eq:inequality_at_colloc}
\end{equation}
Since the mixed inequalities explicitly contain the intermediate values $\tildeyvec_{m+1}^{(1)}$, we need to find a simple expression for them.
Formally, they are introduced as
\begin{align*}
\tildeyvec_{m+1}^{(1)} &= \yvec_m + \Delta_t \cdot \frac{1}{2} \cdot \fvec (t_{m+1/2}^{\mathrm{s}}, \tildeyvec_{m+1}^{(1)}, \uvec_{m+1/2})
  = \yvec_m + \Delta_t \cdot \frac{1}{2} \cdot \Smat_2 \cdot \uvec_{m+1/2}\,.
\\
\intertext{At the same time, we get from the discretization of the dynamic equations}
\yvec_{m+1} &= \yvec_m + \Delta_t \cdot \Smat_2 \cdot \uvec_{m+1/2}\,,
\end{align*}
which allows us to eliminate the control variables, giving
\[
\tildeyvec_{m+1}^{(1)} = \frac{\yvec_m + \yvec_{m+1}}{2}\,~(m = 0,1,\ldots,N-1)\,.
\]
(This is exactly the collocation property of the midpoint rule.)
So, \eqref{eq:inequality_at_colloc} can be simplified in case of the midpoint rule to
\[
\nullvec \leq \gvec_{\leq} \left(t_{m+1/2}^{\mathrm{s}}, \frac{\yvec_m+\yvec_{m+1}}{2}, \uvec_{m+1/2} \right)\,,~(m=0,1,\ldots,N-1).
\]
For the given linear structure of the mixed constraints \eqref{eq:OC_problem:mixed}, this reads
\[
\Hmat_{\yvec} \cdot \frac{\yvec_m+\yvec_{m+1}}{2}  + \Hmat_{\uvec}\cdot  \uvec_{m+1/2} \leq \hvec
\]
and can, once again, be restated as
\[
\frac{1}{2} \cdot
\begin{pmatrix}
	\Hmat_{\yvec} & \Hmat_{\yvec} & \Nullmat      & \cdots        & \Nullmat\\
	\Nullmat      & \Hmat_{\yvec} & \Hmat_{\yvec} & \cdots        & \Nullmat\\
	\vdots        &               & \ddots        & \ddots        & \vdots \\
	\Nullmat      & \cdots        & \cdots        & \Hmat_{\yvec} & \Hmat_{\yvec}
\end{pmatrix}
\cdot 
\begin{pmatrix} 	\yvec_0 \\ \yvec_1 \\ \vdots \\ \yvec_N \end{pmatrix}
+
\begin{pmatrix}
	\Hmat_{\uvec} & \Nullmat      & \cdots & \Nullmat \\
	\Nullmat      & \Hmat_{\uvec} &        & \Nullmat \\
	\vdots        &               & \ddots & \vdots   \\
	\Nullmat      & \cdots        & \cdots & \Hmat_{\uvec}
\end{pmatrix}
\cdot 
\begin{pmatrix}
	\uvec_{1/2} \\ \uvec_{1+1/2} \\ \vdots \\ \uvec_{(N-1)+1/2}
\end{pmatrix}
\leq
\begin{pmatrix} \hvec \\\hvec \\ \vdots \\\hvec \end{pmatrix}
\]
or in Kronecker form
\[
\underbrace{
\frac{1}{2} \cdot \begin{pmatrix}
	1       & 1      & 0      & \cdots  & 0 \\
	0       & 1      & 1      & \cdots  & 0 \\
	\vdots  &        & \ddots & \ddots  & \vdots \\
	0       & \cdots & \cdots &  1      & 1
\end{pmatrix}
\kron \Hmat_{\yvec}
}_{\Afrakmat_{\yvec,\mathrm{mix}}}
\cdot
\underbrace{
\begin{pmatrix}
	\yvec_0 \\ \yvec_1 \\ \vdots \\ \yvec_N
\end{pmatrix}
}_{\yfrakvec}
+
\underbrace{
\Imat_N \kron \Hmat_{\uvec}}_{\Afrakmat_{\uvec,\mathrm{mix}}}
\cdot 
\underbrace{
\begin{pmatrix}
	\uvec_{1/2} \\ \uvec_{1+1/2} \\ \vdots \\ \uvec_{(N-1)+1/2}
\end{pmatrix}
}_{\ufrakvec}
\leq 
\underbrace{
\begin{pmatrix}
	\hvec \\ \hvec\\ \vdots \\ \hvec
\end{pmatrix}
}_{\bfrakvec_{\mathrm{mix}}} \,.
\]


% ______________________________________________________________________________
\subsubsection*{Positivity of Dynamic Variables}

Requiring that the dynamic variables are positive at all time steps is simple. 
We just implement them as lower bounds. 
(Technically speaking, we would require positivity again for the intermediate values only but this is again equivalent.)
\[
\lbfrakvec_{\yvec} = 
\begin{pmatrix} \nullvec_{n_{\yvec}}^{\tp} & \nullvec_{n_{\yvec}}^{\tp} & \cdots & \nullvec_{n_{\yvec}}^{\tp} \end{pmatrix}^{\tp}\,,\quad
\ubfrakvec_{\yvec}=
\begin{pmatrix} \infty & \infty & \cdots & \infty	 \end{pmatrix}^{\tp}\,.
\]


% ______________________________________________________________________________
\subsubsection*{Boundary Conditions}

Imposing the boundary constraints \eqref{eq:OC_problem:bndry} requires only to couple the initial and end approximations of the dynamical variables:
\[
\Bmat_{\yvec_0} \cdot \yvec_0 + \Bmat_{\yvec_{\mathrm{end}}} \cdot \yvec_N = \bvec_{\mathrm{bndry}}\,.
\]
For the LP form, this can be written in (large) matrix form
\[
\underbrace{
\begin{pmatrix} \Bmat_{\yvec_0} & \Nullmat & \cdots & \Nullmat & \Bmat_{\yvec_{\mathrm{end}}}\end{pmatrix}
}_{\Afrakmat_{\yvec,\mathrm{bndry}}}
\cdot
\underbrace{
\begin{pmatrix}
	\yvec_0 \\ \yvec_1 \\ \vdots \\ \yvec_N
\end{pmatrix}
}_{\yfrakvec}
+
\underbrace{\Nullmat}_{\Afrakmat_{\uvec,\mathrm{bndry}}}
\cdot 
\underbrace{
\begin{pmatrix}
	\uvec_{1/2} \\ \uvec_{1+1/2} \\ \vdots \\ \uvec_{(N-1)+1/2}
\end{pmatrix}
}_{\ufrakvec}
=
\bvec_{\mathrm{bndry}}
\]


% ______________________________________________________________________________
\subsubsection*{Putting it all together}

For the final formulation of the LP, we collect all the information like
\begin{align*}
\ffrakvec &= \begin{pmatrix} \ffrakvec_{\yvec}^{\tp} & \nullvec^{\tp} \end{pmatrix}^{\tp}\,,
\\
\Afrakmat_{\leq} &= \begin{pmatrix} \Afrakmat_{\yvec,\mathrm{mix}}, \Afrakmat_{\uvec,\mathrm{mix}} \end{pmatrix}\,,
\\
\bfrakvec_{\leq} &= \bfrakvec_{\mathrm{mix}}
\\
\Afrakmat_{=} &= \begin{pmatrix} 
                     \Afrakmat_{\yvec, \mathrm{dyn}} & \Afrakmat_{\uvec, \mathrm{dyn}} \\
                     \Afrakmat_{\yvec, \mathrm{qssa}} & \Afrakmat_{\uvec, \mathrm{qssa}} \\
                     \Afrakmat_{\yvec, \mathrm{bndry}} & \Afrakmat_{\uvec, \mathrm{bndry}}
                 \end{pmatrix}
\\
\bfrakvec_{=} &= \begin{pmatrix} \nullvec^{\tp} & \nullvec^{\tp} & \bvec_{\mathrm{bndry}}^{\tp}\end{pmatrix}^{\tp}
\\
\lbfrakvec &= \begin{pmatrix} \lbfrakvec_{\yvec}^{\tp} & \lbfrakvec_{\uvec}^{\tp} \end{pmatrix}^{\tp}
\\
\ubfrakvec &= \begin{pmatrix} \ubfrakvec_{\yvec}^{\tp} & \ubfrakvec_{\uvec}^{\tp} \end{pmatrix}^{\tp}\,.
\end{align*}




% %%%%%%%%%%%%%%%%%%%%%%%%%%%%%%%%%%%%%%%%%%%%%%%%%%%%%%%%%%%%%%%%%%%%%%%%%%%%%%
\appendix

\subsection*{ToDos and Maybe's}

\begin{itemize}
  \item time-dependent matrices on almost all levels
	\item more general time integration schemes
	\item additional linear and constant terms in the dynamics and the control constraints (which would become mixed constraints then as well)
	\item non-equidistant time grid
	\item Add terms depending on controls in the objective and the boundary conditions
	\item add quadratic terms in the objective
	\item inequality mixed constraints/bounds on dynamic variables(?)
\end{itemize}


\end{document}
