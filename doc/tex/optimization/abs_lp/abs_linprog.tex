\documentclass{article}


\usepackage[utf8]{inputenc}
\usepackage[T1]{fontenc}
\usepackage[USenglish]{babel}
\usepackage{amsmath,amssymb}
\usepackage{fullpage}
\usepackage{xcolor}
\usepackage{nicematrix}
%\usepackage{todonotes}


\makeatletter
\providecommand*{\diff}%
   {\@ifnextchar^{\DIfF}{\DIfF^{}}}
\def\DIfF^#1{%
   \mathop{\text{\mathstrut d}}%
      \nolimits^{#1}\gobblespace}
\def\gobblespace{%
      \futurelet\diffarg\opspace}
\def\opspace{%
   \let\DiffSpace\!%
   \ifx\diffarg(%
      \let\DiffSpace\relax
   \else
      \ifx\diffarg[%
         \let\DiffSpace\relax
      \else
         \ifx\diffarg\{%
            \let\DiffSpace\relax
         \fi\fi\fi\DiffSpace}
\makeatother
\newcommand{\intd}{\diff}%             differential "d"
\newcommand{\defeq}{\mathrel{:=}}%     is per definition equal to
\newcommand{\tp}{\top}%                transpose
\newcommand{\kron}{\otimes}%           Kronecker product
%\newcommand{\dkron}[2][]{\mathbin{{}_{#1}\negthinspace\kron_{#2}}}% generalized Kronecker product
\newcommand{\dkron}[2][]{\mathbin{\kron_{#2}^{#1}}}% generalized Kronecker product
\newcommand{\setR}{\mathbb{R}}%        real numbers
\newcommand{\expe}{\mathrm{e}}%        Eulers constant
\newcommand{\diag}{\mathop{\mathrm{diag}}}%     diagonal elements of matrix/diagonal matrix from vector


\newcommand{\vectorfont}[1]{\boldsymbol{#1}}%
\newcommand{\greekvectorfont}[1]{\boldsymbol{#1}}%
\newcommand{\matrixfont}[1]{\mathbf{#1}}%


\newcommand{\bvec}{\vectorfont{b}}
\newcommand{\fvec}{\vectorfont{f}}
\newcommand{\gvec}{\vectorfont{g}}
\newcommand{\hvec}{\vectorfont{h}}
\newcommand{\kvec}{\vectorfont{k}}
\newcommand{\nvec}{\vectorfont{n}}
\newcommand{\tvec}{\vectorfont{t}}
\newcommand{\uvec}{\vectorfont{u}}
\newcommand{\xvec}{\vectorfont{x}}
\newcommand{\yvec}{\vectorfont{y}}

\newcommand{\tildeyvec}{\vectorfont{\tilde{y}}}

\newcommand{\dyvec}{\vectorfont{\dot{y}}}
%
\newcommand{\Deltavec}{\greekvectorfont{\Delta}}
\newcommand{\Phivec}{\greekvectorfont{\Phi}}
\newcommand{\Psivec}{\greekvectorfont{\Psi}}
%
\newcommand{\bfrakvec}{\vectorfont{\mathfrak{b}}}
\newcommand{\ffrakvec}{\vectorfont{\mathfrak{f}}}
\newcommand{\kfrakvec}{\vectorfont{\mathfrak{k}}}
\newcommand{\ufrakvec}{\vectorfont{\mathfrak{u}}}
\newcommand{\xfrakvec}{\vectorfont{\mathfrak{x}}}
\newcommand{\yfrakvec}{\vectorfont{\mathfrak{y}}}
%
\newcommand{\tildeyfrakvec}{\vectorfont{\tilde{\mathfrak{y}}}}
\newcommand{\nullvec}{\greekvectorfont{0}}
\newcommand{\lbvec}{\vectorfont{l\negthinspace b}}
\newcommand{\ubvec}{\vectorfont{u\negthinspace b}}
\newcommand{\lbfrakvec}{\vectorfont{\mathfrak{l\negthinspace b}}}
\newcommand{\ubfrakvec}{\vectorfont{\mathfrak{u\negthinspace b}}}
\newcommand{\einsvec}{\vectorfont{1}\negthinspace\negthinspace\vectorfont{1}} % vector of only ones (1,1,...,1)^T
%
\newcommand{\Amat}{\matrixfont{A}}
\newcommand{\Bmat}{\matrixfont{B}}
\newcommand{\Hmat}{\matrixfont{H}}
\newcommand{\Imat}{\matrixfont{I}}%       identity matrix
\newcommand{\Mmat}{\matrixfont{M}}
\newcommand{\Smat}{\matrixfont{S}}
\newcommand{\Tmat}{\matrixfont{T}}

\newcommand{\Afrakmat}{\matrixfont{\mathfrak{A}}}

\newcommand{\Nullmat}{\matrixfont{0}}

% https://tex.stackexchange.com/questions/121955/help-on-dealing-with-items-divided-with-slash
% latex.ltx, line 467:
%\def\slash{/\penalty\exhyphenpenalty} % a `/' that acts like a `-'
\renewcommand{\slash}{/\penalty\exhyphenpenalty\hspace{0pt}}
%\makeatletter
%\def\slash{/\penalty\z@\hskip\z@skip }
%\makeataother



% http://lpsolve.sourceforge.net/5.1/absolute.htm for a more detailed explanation of most steps

\begin{document}

\begin{center}
	{\Large\textbf{Implementation Details for the \texttt{MinabsLPModel} class}}
\end{center}

Starting with an absolute value linear problem of the form
\begin{subequations}
\begin{align}
\min_{\xvec} \;& \;\fvec_1^{\tp} \cdot \lvert \Mmat_f \cdot \xvec - \nvec_f \rvert + \fvec_2^{\tp} \cdot \xvec
\label{eq:minabs:obj}
\\
\text{s.\,t.} \quad
\Amat_{=}\cdot \xvec &= \bvec_{=}
\label{eq:minabs:eq1}
\\
\Amat_{\leq} \cdot \xvec &\leq \bvec_{\leq}
\label{eq:minabs:ineq1}
\\
\lbvec &\leq \xvec \leq \ubvec
\label{eq:minabs:bounds}
\\
\Mmat_{c,1} \cdot \lvert \Mmat_{c,2} \cdot \xvec -\nvec_c \rvert + \Mmat_{c,3}\cdot \xvec &\leq \bvec_c
\label{eq:minabs:absineq}
%\\
\end{align}
\end{subequations}
with
\[
\begin{split}
\fvec_1 &\in \setR^{m_f}_{\geq 0},~
\Mmat_f \in \setR^{m_f \times n_{\xvec}},~
\nvec_f \in \setR^{m_f},~
\fvec_2 \in \setR^{n_{\xvec}},
\\
\Amat_{=} &\in \setR^{m_= \times n_{\xvec}},~
\bvec_{=} \in \setR^{m_=},~
\Amat_{\leq} \in \setR^{m_{\leq}\times n_{\xvec}},~~
\bvec_{\leq} \in \setR^{m_\leq},~
\lbvec \in \setR^{n_{\xvec}},~
\ubvec \in \setR^{n_{\xvec}},
\\
\Mmat_{c,1} &\in \setR^{m_c \times m_1}_{\geq 0},~
\Mmat_{c,2} \in \setR^{m_1 \times n_{\xvec}},~
\nvec_c \in \setR^{m_1},~
\Mmat_{c,3} \in \setR^{m_c \times n_{\xvec}},~
\bvec_c \in \setR^{m_c}\,.
%\label{eq:minabs:dimen} 
\end{split}
\]
and where the absolute value is to be understood component-wise.

The idea is fairly standard:
\begin{enumerate}
  \item{\color{gray}Sort out those rows $(\Mmat_f \cdot \xvec)_i$ with $(\fvec)_i = 0$ as this does nothing to the objective.
  Sort also the zeros rows of $(\Mmat_{c,2},-\nvec_c)$ and the zero columns of $\Mmat_{c,1}$ out.
  If one $(\bvec_c)_i < 0$, the problem is infeasible and there is nothing to be done.
  Zero rows in $\Mmat_{c,0}$ lead to regular inequality constraints
  }
  \item{All the rows $(\Mmat_c \cdot \xvec - \nvec_c)_i$ with $(\bvec_c)_i = 0$ can simply be replaced by an equality constraint, all those with $(\bvec)_i > 0$ by two inequalities.
  }
  \item{Introduce new variables $\xvec_{\mathrm{new},f} \in \setR^{m_f^\ast}$ to treat the remaining rows of $\Mmat_{f} \cdot \xvec - \nvec_{f}$.
  Equivalently, introduce new variables $\xvec_{\mathrm{new},c} \in \setR^{m_1^\ast}$ to treat the abs-parts in the constraints.
  }
  \item{Bound the new variables appropriately.
  }
  % DISUSED BUT MAYBE RE-USABLE LATER
  %\item{Use the positive and negative parts of the new variables to handle the absolute values: For a real-valued variable $x$, this reads like:
  %	$\lvert x \rvert = x_+ + x_-$, $x_+,\, x_- \geq 0$, $x = x_+ - x_-$.
  %}
\end{enumerate}

%\paragraph{Variable Names}

\paragraph{LP Model Setup}

The entire LP can be formulated as
\begin{align*}
\min_{\xfrakvec}\, &\,\ffrakvec^{\tp} \cdot \xfrakvec 
\\
\text{s.\,t. }
\Afrakmat_{\leq} \cdot \xfrakvec &\leq \bfrakvec_{\leq}
\\
\Afrakmat_{=} \cdot \xfrakvec &= \bfrakvec_{=}
\\
\lbfrakvec &\leq \xfrakvec \leq \ubfrakvec
\end{align*}
with
\NiceMatrixOptions{code-for-first-row = \color{gray}}
\begin{align*}
\xfrakvec &\defeq (\xvec^{\tp}, \xvec_{\mathrm{new},f}^{\tp}, \xvec_{\mathrm{new},c}^{\tp})^{\tp}
&
\ffrakvec &\defeq (\fvec_2^{\tp}, \fvec_1^{\tp}, \nullvec_{m_c^\ast}^{\tp})^{\tp}
\\
\Afrakmat_{\leq} &\defeq 
\begin{pNiceMatrix}[first-row]%[first-row,last-row,first-col,last-col]
\xvec   & \xvec_{\mathrm{new},f} & \xvec_{\mathrm{new},c} \\
\Amat_{\leq} & \Nullmat          & \Nullmat \\
\Mmat_f      & -\Imat_{m_f^\ast} & \Nullmat \\
-\Mmat_f     & -\Imat_{m_f^\ast} & \Nullmat \\
\Mmat_{c,2}  & \Nullmat          & -\Imat_{m_1^\ast} \\
-\Mmat_{c,2} & \Nullmat          & -\Imat_{m_1^\ast} \\
\Mmat_{c,3}  & \Nullmat          & \Mmat_{c,1}
\end{pNiceMatrix}\,,
&
\bfrakvec_{\leq} &\defeq 
\begin{pmatrix} \bvec_{\leq} \\ \nvec_f \\ - \nvec_f \\ \nvec_c \\ -\nvec_c \\ \bvec_c
\end{pmatrix}
\\
\Afrakmat_{=} &\defeq 
\begin{pNiceMatrix}[first-row]%[first-row,last-row,first-col,last-col]
\xvec   & \xvec_{\mathrm{new},f} & \xvec_{\mathrm{new},c} \\
\Amat_= & \Nullmat & \Nullmat
\end{pNiceMatrix}\,,
&
\bfrakvec_{=} &\defeq \bvec_=
\\
\lbfrakvec &\defeq \begin{pmatrix}\lbvec \\ \nullvec \end{pmatrix}\,,&
\ubfrakvec &\defeq \begin{pmatrix} \ubvec \\ \infty \cdot \einsvec\end{pmatrix}
\end{align*}

\appendix

\subsection*{ToDos\slash Questions and Maybe's}

\begin{itemize}
	\item uniquify the rows before the introduction of $\xvec_{\mathrm{new}}$? How can we uniquely translate back?
	\item What about ILP\slash MILP and\slash or indicator variables?
	\item Is there any special treatment of the \texttt{variable\_names} necessary?
	\item Any special considerations when extending to a min\slash max problem?
	\item Equality constraints with absolute values?
	\item The upper bounds in the big model can be set lower if the according upper bounds on $\xvec$ are known. Whether this makes a difference, is debatable...
	\item Benchmarks, tests and\slash or Documentation
	%\item We should for more flexibility replace \eqref{eq:minabs:absineq} by
	%\[
	%\Mmat_{c,0} \cdot \lvert \Mmat_{c,1} \cdot \xvec - \nvec_{c,1} \rvert - \Mmat_{c,2}\cdot \lvert \Mmat_{c,3}\cdot \xvec - \nvec_{c,2} \rvert + %\Mmat_{c,4} \cdot \xvec \leq \bvec_c.
	%\]
	\item Why do all $(\fvec_1)_i$, $(\Mmat_{c,0})_{i,j}$ have to be positive? ($\to$ binary variables\slash special variables ) 
	\item\texttt{None}-handler: If the according variable is \lq uninteresting\rq, we can replace it in the input with \texttt{None}.
\end{itemize}


\end{document}







The entire LP with $x_+$, $x_-$ can be formulated as
\begin{align*}
\min_{\xfrakvec}\, &\,\ffrakvec^{\tp} \cdot \xfrakvec 
\\
\text{s.\,t. }
\Afrakmat_{\leq} \cdot \xfrakvec &\leq \bfrakvec_{\leq}
\\
\Afrakmat_{=} \cdot \xfrakvec &= \bfrakvec_{=}
\\
\lbfrakvec &\leq \xfrakvec \leq \ubfrakvec
\end{align*}
with
\NiceMatrixOptions{code-for-first-row = \color{gray}}
\begin{align*}
\xfrakvec &\defeq (\xvec^{\tp}, \xvec_{\mathrm{new}}^{\tp}, \xvec_{\mathrm{new},+}^{\tp}, \xvec_{\mathrm{new},-}^{\tp})^{\tp}
&
\ffrakvec &\defeq (\fvec_2^{\tp}, \fvec_1^{\tp}, \nullvec_{m_c + 2 \cdot (m_f+m_c)}^{\tp})^{\tp}
\\
\Afrakmat_{\leq} &\defeq
\begin{pmatrix} 
\Amat_{\leq} &                     &            & \\
& \Nullmat_{m_c, m_f} &\Imat_{m_c} & \Nullmat_{m_c, 2\cdot (m_c+m_f)}
\end{pmatrix}\,,
&
\bfrakvec_{\leq} &\defeq \begin{pmatrix} \bvec_{\leq}\\ \bvec_c \end{pmatrix}\,,
\\
\Afrakmat_{=} &\defeq 
\begin{pNiceMatrix}[first-row]%[first-row,last-row,first-col,last-col]
\xvec   & \xvec_{\mathrm{new}} & \xvec_{\mathrm{new},+} & \xvec_{\mathrm{new},-} \\
\Amat_= &                      &                        &                        \\
\begin{array}{c}\ensuremath \Mmat_f \\ \ensuremath \Mmat_c\end{array}
& \begin{array}{cc}-\Imat_{m_f} & \\ &-\Imat_{m_c}\end{array}      
&                        &                        \\
& \Imat_{m_f+m_c}      & -\Imat_{m_f+m_c}       & \Imat_{m_f+m_c}   
\end{pNiceMatrix}\,,
&
\bfrakvec_{=} &\defeq
\begin{pmatrix}
\bvec_{=} \\ \begin{array}{c} \ensuremath \nvec_f \\\ensuremath \nvec_c\end{array} \\ \nullvec_{m_f+m_c}
\end{pmatrix}
\\
\lbfrakvec &\defeq (\lbvec^{\tp}, -\infty \cdot \einsvec_{m_f+m_c}^{\tp}, \nullvec_{2 \cdot (m_f+m_c)}^{\tp})^{\tp} 
&
\ubfrakvec &\defeq (\ubvec^{\tp}, \infty\cdot \einsvec_{3 \cdot (m_f+m_c)}^{\tp})^{\tp}\,.
\end{align*}
